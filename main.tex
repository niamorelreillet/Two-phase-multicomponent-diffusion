\documentclass[3p]{elsarticle}
\usepackage{graphicx,pst-all}
\usepackage{epstopdf}
\usepackage{multirow}
\usepackage{xspace}
\usepackage{tabularx}
\usepackage{rotating}
\usepackage{float}
\usepackage{amsmath,amssymb,pifont}
\usepackage[mathscr]{eucal}
\usepackage[english,frenchb]{babel}
\usepackage[T1]{fontenc}
\usepackage[utf8]{inputenc}

\newcommand{\n}{\tabularnewline}
\newcolumntype{C}{>{\centering}X}
\newcolumntype{L}{>{\raggedleft}X} 
\newcolumntype{R}{>{\raggedright}X} 
\newcolumntype{M}[1]{>{\centering}m{#1}}

\newenvironment{legend}%
 {\tabular{r>{\small} l}}%
 {\endtabular}

\newcommand{\Rm}[1]{{\bf R}_{#1}^{g+}}
\newcommand{\Rr}[1]{{\bf r}_{#1}^{g\ast}}
\newcommand{\Sm}[1]{{S_{#1}^{g+}}}
\newcommand{\Sr}[1]{{s_{#1}^{g\ast}}}
\newcommand{\Nr}[1]{N_{#1}^{\ast}}
\newcommand{\M}{\mathbb}
\newcommand{\B}[1]{{\bf #1}\xspace}
\newcommand{\T}{\texttt}
\newcommand{\Mc}{\mathcal}
\newcommand{\Hi}{\textit}
\newcommand{\norm}[1]{\left\|#1\right\|}
\newcommand{\abs}[1]{\left|#1\right|}
\newcommand{\snorm}[1]{\|#1\|}
\newcommand{\psca}[2]{\left\langle #1 , #2 \right\rangle}
\newcommand{\pscac}[2]{\textcolor{red}{\left\langle \textcolor{black}{#1} , \textcolor{black}{#2} \right\rangle}}
\newcommand{\la}{\Big\langle}
\newcommand{\ra}{\Big\rangle}
\newcommand{\im}{{\bf i}\xspace}
\newcommand{\sg}{\textrm{sg}}

\newcommand{\Eq}[1]{Eq.~\ref{eq:#1}}
\newcommand{\Eqs}[2]{Eqs.~\ref{eq:#1} and~\ref{eq:#2}}
\newcommand{\Eqss}[3]{Eqs.~\ref{eq:#1},~\ref{eq:#2} and~\ref{eq:#3}}
\newcommand{\Fig}[1]{Figure~\ref{fig:#1}}
\newcommand{\Figs}[2]{Figures~\ref{fig:#1} and~\ref{fig:#2}}
\newcommand{\Figss}[3]{Figures~\ref{fig:#1},~\ref{fig:#2} and~\ref{fig:#3}}
\newcommand{\Tab}[1]{Table~\ref{tab:#1}}
\newcommand{\Tabs}[2]{Tables~\ref{tab:#1} and~\ref{tab:#2}}
\newcommand{\Tabss}[3]{Tables~\ref{tab:#1},~\ref{tab:#2} and~\ref{tab:#3}}
\newcommand{\Sect}[1]{Section~\ref{sect:#1}}
\newcommand{\Sects}[2]{Sections~\ref{sect:#1}~and~\ref{sect:#2}}
\newcommand{\Sec}[1]{\Sect{#1}}
\newcommand{\App}[1]{Appendix~\ref{app:#1}}
%\renewcommand{\emph}[1]{\textcolor{red}{\bf #1}}

% notations
\newcommand{\functional}[2][\cdot]{\mathbb{#2}\left[#1\right]}
\newcommand{\function}[2][\cdot]{#2\left(#1\right)}
\newcommand{\grad}[1]{\nabla #1}
\newcommand{\dX}[1]{\frac{\partial #1}{\partial x}}
\renewcommand{\div}[1]{\nabla \cdot #1}
\newcommand{\Lapl}[1]{\Delta #1}
\newcommand{\LaplX}[1]{\frac{\partial^2 #1}{\partial x^2}}
\newcommand{\domain}[1]{V_{#1}}
\newcommand{\boundary}[1]{\partial \domain{#1}}
\newcommand{\interface}[1]{\Gamma_{#1}}
\newcommand{\MatrixIJ}[3]{\left[#1\right]_{#2,#3}}
\DeclareMathOperator*{\argmin}{arg\,min}

\newenvironment{abstracts}
 {\global\setbox\absbox=\vbox\bgroup
    \hsize=\textwidth
    \linespread{1}\selectfont}
 {\vspace{-\bigskipamount}\egroup}
\renewenvironment{abstract}[1][]
 {\if\relax\detokenize{#1}\relax\else\selectlanguage{#1}\fi
  \noindent\textbf{\abstractname}\par\medskip\noindent\ignorespaces}
 {\par\bigskip}

\raggedbottom

\makeatletter
\def\ps@pprintTitle{
      \let\@oddhead\@empty
      \let\@evenhead\@empty
      \def\@oddfoot{\reset@font\hfil\thepage\hfil}
      \let\@evenfoot\@oddfoot
}
\makeatother
\bibliographystyle{elsarticle-num}

\begin{document}

\begin{frontmatter}

\title{Working document on two-phase multicomponent modelling}
% 
% \author[Cadarache]{R.~Le Tellier\corref{contact}}
% \ead{romain.le-tellier@cea.fr}
% \cortext[contact]{contact:}
% 
% \address[Cadarache]{CEA, DEN, DTN/SMTA/LMAG, Cadarache \\
%   F-13108 Saint Paul-lez-Durance, France}
  
% \begin{abstracts}
% \selectlanguage{frenchb}
% \begin{abstract}[french]
% %   Resum\'e en fran\c{c}ais
% 
%  \end{abstract}
% \selectlanguage{english}
% \begin{abstract}
% %   Abstract in English 
% \end{abstract}
% \end{abstracts}
  
\end{frontmatter}

\selectlanguage{frenchb}
\section{Basic equations}

\subsection{Multicomponent diffusion equation}

\subsubsection{Written in terms of components (elements)}

In a $n-$component system, the mass conservation law is expressed as:
\begin{equation}
 \frac{\partial c_i}{\partial t} = -\div{J_i} \label{eq:ci}
\end{equation}
where $c_i$ is the concentration of component $i$ and $J_i$ is the molar flux of component $i$.

In the generalized linear diffusion theory for $n-$component systems (see \cite{Andersson1992}), for a given phase, $J_i$ is expressed as a linear combination of the concentration gradients:
\begin{equation}
J_i = - \sum_{j=1}^n D_{ij} \grad(c_j) \label{eq:Ji}
\end{equation}
where $\MatrixIJ{D_{ij}}{i}{j}$ is the diffusion coefficient matrix.

$\MatrixIJ{D_{ij}}{i}{j}$ is evaluated under the form:
\begin{equation}
 D_{ij} =  \sum_{k=1}^n \left(\delta_{ki} -\frac{V_k}{V_m} x_i\right) x_k \frac{D_k^\star}{RT} \frac{\partial \mu_k}{\partial x_j} 
\end{equation}
where 
\begin{itemize}
 \item $\delta_{ki}$ is the Kronecker delta;
 \item $x_i$ is the mole fraction of component $i$;
 \item $V_m = \sum_{i} x_i V_i$ is the molar volume;
 \item $D_k^\star$ is the self-diffusion coefficient of component $k$;
 \item $\frac{\partial \mu_k}{\partial x_j}$ are the derivatives of the chemical potentials.
\end{itemize}
Note that, in the general case, $\frac{\partial \mu_k}{\partial x_j}$ quantities makes only sense under a \emph{local} equilibrium hypothesis (in the corium system case, corresponding to the equilibrium of the redox reactions for a given component inventory). Such terms cannot be evaluated directly and have to be calculated from separated equilibrium calculations through a finite difference formula.

\subsubsection{Written in terms of endmembers (species)}

Associated with the $n-$component system, the CALPHAD based representation of the system involves a a set of $m$ endmembers that, in the case of the \T{LIQUID} and \T{C1\_FCC} phases of the NUCLEA database, are stoichiometric species and the constituents of the non-ideal associate model that is used for representing these phases. 

Equations similar to \Eqs{ci}{Ji} can also be written in terms of endmembers and, in this case, the associated diffusion coefficient matrix is
\begin{equation}
 \tilde{D}_{ij} =  \sum_{k=1}^m \left(\delta_{ki} -\frac{V_k}{V_m} x_i\right) x_k \frac{\tilde{D}_k^\star}{RT} \frac{\partial \mu_k}{\partial x_j} 
\end{equation}
where 
\begin{itemize}
 \item $\delta_{ki}$ is the Kronecker delta;
 \item $x_i$ is the mole fraction of endmember $i$;
 \item $V_m = \sum_{i} x_i V_i$ is the molar volume;
 \item $\tilde{D}_k^\star$ is the self-diffusion coefficient of endmember $k$;
 \item $\frac{\partial G_k}{\partial x_j}$ are the derivatives of the partial Gibbs energy w.r.t. endmembers.
\end{itemize}
In principle, $\frac{\partial G_k}{\partial x_j}$ quantities can be directly evaluated from the database.

\subsection{Interface condition}

\subsubsection{Dissolution}

We consider an isothermal two-phase system with a solid phase (superscript $s$) being dissoluted by the liquid one (superscript $l$).

The jump interface condition for component $i$ concentration are written as:
\begin{equation}
v^{l/s} \left(c_i^l-c_i^s\right) = J_i^l - J_i^s
\end{equation}
Assuming local equilibrium at the interface, $x_i^l$ and $x_i^s$ are the concentrations associated with a liquid-solid tie line. 
Under fixed pressure and temperature, these equations defines a system of $n-2$ non-linear equations whose unknowns are $v^{l/s}$ and $n-2$ chemical potentials.

\subsubsection{Oxidation}

We consider an isothermal two-phase system with a liquid phase (superscript $l$) being oxidized by the gas one (superscript $g$).

\emph{to be completed}

\section*{References}

\bibliography{../../../../write/References/lma-jabref.bib}

\end{document}